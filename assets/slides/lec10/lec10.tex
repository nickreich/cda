\documentclass[ignorenonframetext,]{beamer}
\setbeamertemplate{caption}[numbered]
\setbeamertemplate{caption label separator}{: }
\setbeamercolor{caption name}{fg=normal text.fg}
\beamertemplatenavigationsymbolsempty
\usepackage{lmodern}
\usepackage{amssymb,amsmath}
\usepackage{ifxetex,ifluatex}
\usepackage{fixltx2e} % provides \textsubscript
\ifnum 0\ifxetex 1\fi\ifluatex 1\fi=0 % if pdftex
  \usepackage[T1]{fontenc}
  \usepackage[utf8]{inputenc}
\else % if luatex or xelatex
  \ifxetex
    \usepackage{mathspec}
  \else
    \usepackage{fontspec}
  \fi
  \defaultfontfeatures{Ligatures=TeX,Scale=MatchLowercase}
\fi
% use upquote if available, for straight quotes in verbatim environments
\IfFileExists{upquote.sty}{\usepackage{upquote}}{}
% use microtype if available
\IfFileExists{microtype.sty}{%
\usepackage{microtype}
\UseMicrotypeSet[protrusion]{basicmath} % disable protrusion for tt fonts
}{}
\newif\ifbibliography
\hypersetup{
            pdftitle={Lecture 10: GLMs: Poisson Regression, Overdispersion},
            pdfauthor={Nick Reich / Transcribed by Daveed Goldenberg, edited by Josh Nugent},
            pdfborder={0 0 0},
            breaklinks=true}
\urlstyle{same}  % don't use monospace font for urls

% Prevent slide breaks in the middle of a paragraph:
\widowpenalties 1 10000
\raggedbottom

\AtBeginPart{
  \let\insertpartnumber\relax
  \let\partname\relax
  \frame{\partpage}
}
\AtBeginSection{
  \ifbibliography
  \else
    \let\insertsectionnumber\relax
    \let\sectionname\relax
    \frame{\sectionpage}
  \fi
}
\AtBeginSubsection{
  \let\insertsubsectionnumber\relax
  \let\subsectionname\relax
  \frame{\subsectionpage}
}

\setlength{\parindent}{0pt}
\setlength{\parskip}{6pt plus 2pt minus 1pt}
\setlength{\emergencystretch}{3em}  % prevent overfull lines
\providecommand{\tightlist}{%
  \setlength{\itemsep}{0pt}\setlength{\parskip}{0pt}}
\setcounter{secnumdepth}{0}
%       ************************************************
%       **        LaTeX preamble to be used with all 
%	**        statsTeachR labs/handouts.
%
%	Author: Nicholas G Reich
%	Last modified: July 2017
%	************************************************

%\documentclass[table]{beamer}

%	Set theme (a nice plain one)
\usetheme{Malmoe}

%	Use named colors, set main color of theme
%		to match Web site color:
\definecolor{MainColor}{RGB}{10, 74, 109}
\colorlet{MainColorMedium}{MainColor!50}
\colorlet{MainColorLight}{MainColor!20}
\usecolortheme[named=MainColor]{structure} 

%	For tables
%[dvipsnames] [table]
\usepackage{xcolor}

%% calling tabu.sty, assuming a particular directory structure
\usepackage{../../slide-includes/tabu}	% Even fancier than tabulary
\usepackage{multirow}

%	Just for the degree symbol
\usepackage{textcomp}

%	Get rid of footline (page, author, etc. on each slide)
\setbeamertemplate{footline}{}
%	Get rid of navigation buttons
\setbeamertemplate{navigation symbols}{}

%	Make footnotes not ugly
\usepackage{hanging}
\setbeamertemplate{footnote}{\raggedright\hangpara{1em}{1}\makebox[1em][l]{\insertfootnotemark}\footnotesize\insertfootnotetext\par}

%	Text style for code snippets inline in text:
\newcommand{\codeInline}[1]{\texttt{#1}}

%	Text style for emphasis stronger than \emph:
%		(Note, this doesn't toggle the way \emph does.
%			(Note, this can be done, didn't seem worth the trouble.))
\newcommand{\strong}[1]{{\bfseries{#1}}}


%        ******	Define title page	**********************
\setbeamertemplate{title page}{
	{\color{MainColor}
	% There must be a better way than this -vspace at
	%	 the top and bottom of the page to reduce the 
	%	 bottom margin, but I can't find one that works.
	%\vspace{-6em}

% 	% Go to a lot of trouble to get the title in a
% 	%	nice box, since customizing a beamer block
% 	%	does not entirely work here (I don't know why)
	\newlength{\titleBoxWidth}
	\setlength{\titleBoxWidth}{\textwidth}
	\addtolength{\titleBoxWidth}{-2.0em}
	\setlength{\fboxsep}{1.0em}
	\setlength{\fboxrule}{0pt}
	\fcolorbox{MainColor!25}{MainColor!25}{
		\parbox{\titleBoxWidth}{
			\raggedright
			\LARGE\textbf{\inserttitle}
		}	% end parbox
	}	% end fcolorbox

	\vfill
	\small{Author: \insertauthor}
	\vspace{\baselineskip}

	\small{Course: \underline{\href{http://nickreich.github.io/cda}{Categorical Data Analysis}} (BIOSTATS 743)}

%	\small{\Instructor}
%	\vspace{\baselineskip}

%	\small{\emph{This material is part of the \strong{statsTeachR} project}}

	\vfill
	
	\tiny{\emph{Made available under the \underline{\href{http://creativecommons.org/licenses/by-sa/4.0/}{Creative Commons Attribution-ShareAlike 4.0 International License}.}} \hfill \includegraphics[height=1em]{../../slide-includes/by-sa-compact.png}
 }


		\vspace{-15em}

	}	% end color
	\clearpage
}	% end define title page

\input{../../slide-includes/shortcuts}

\hypersetup{colorlinks,linkcolor=,urlcolor=MainColor}

\title{Lecture 10: GLMs: Poisson Regression, Overdispersion}
\author{Nick Reich / Transcribed by Daveed Goldenberg, edited by Josh Nugent}
\date{}

\begin{document}
\frame{\titlepage}

\begin{frame}{Poisson GLMs}

\begin{itemize}
\tightlist
\item
  Imagine you have count data following the Poisson distribution: \[
  Y_i \sim Poisson(\lambda_i)
  \]
\item
  \(Y_i\) is the total count in the time interval, \(\lambda_i\) is
  \(E(Y_i)\), that is, the risk/rate of occurrence in some time
  interval,\\
\item
  We use a log link for our GLM:
\end{itemize}

\[
\eta_i = X_i\beta=\log{\lambda_i}=g(\lambda_i)=g(E[y_i])
\]

\end{frame}

\begin{frame}{Poisson GLMs}

\begin{itemize}
\tightlist
\item
  Key Points:
\item
  Log link implies multiplicative effect of covariates
\end{itemize}

\[
\begin{aligned}
&log(\lambda) = \beta_0 + \beta_1X_1 + \beta_2X_2\\
& \lambda = e^{\beta_0}e^{\beta_1X_1}e^{\beta_2X_2}
\end{aligned}
\] - Relative risk is the interpretation for \(e^{\beta}\)

\[
\begin{aligned}
log(\lambda_i |X_1 = k+1, X_2 = c) = \beta_0 + \beta_1(k+1) + \beta_2(c) \\
- log(\lambda_i | X_1 = k, X_2 = c) = \beta_0 + \beta_1(k) + \beta_2(c) \\
log((\lambda_i |X_1 = k+1, X_2 = c) / \lambda_i |X_1 = k+1, X_2 = c) = \beta_1
\end{aligned}
\]

\end{frame}

\begin{frame}{Exposure / Offset Term}

Often Poisson models have an `exposure' or `offset' term, representing a
demoninator of some kind. Examples: Let \(u_i\) be offset for
\(Y_i\)\ldots{}

\begin{itemize}
\item
  Disease incidence: \(Y_i\) = the number of cases of flu in a
  population in 1 year (in location i), \(u_i\) = population size
\item
  Accident rates: \(Y_i\) = the number of traffic accidents at site i in
  1 day, \(u_i\) = average number of vehicles travelling through site i
  in 1 day, or \(u_i\) = the number of vehicles through site i yesterday
\item
  The offset is used to scale the \(Y_i\)
\end{itemize}

\end{frame}

\begin{frame}{Exposure / Offset Term}

\[
\begin{aligned}
Y_i \sim Poisson(u_i * \lambda_i) \\
E(Y_i) = u_i *  \lambda_i \\
log(E(Y_i)) = log(u_i) + log(\lambda_i)
\end{aligned}
\]

\begin{itemize}
\tightlist
\item
  \(log(u_i)\) is our offset (from observed data, can be thought of as
  an intercept)
\item
  log(\(\lambda_i\)) is our \(\eta_i\) (the linear predictor)
\end{itemize}

\end{frame}

\begin{frame}{Exposure / Offset Term}

\begin{itemize}
\tightlist
\item
  In R, the Poisson glm can be specified with an offset
\item
  glm(Y \(\sim X_1 + X_2\), family = `poison', offset = log(u), data
  \ldots{})
\item
  the log is important in order to get the correct offset
\item
  The offset term is adding more information to the model but not
  estimating a coefficient
\end{itemize}

\end{frame}

\begin{frame}{Exposure / Offset Term}

\[
\begin{aligned}
Y_i \sim Poisson(u_i * \lambda_i)
\end{aligned}
\]

\begin{itemize}
\tightlist
\item
  The \(Y_i\) could be cases per day
\item
  \(u_i\) could be population (persons)
\item
  then \(\lambda_i\) would be cases per day per population (persons)
\item
  which makes this a rate for an individual
\end{itemize}

\[
\begin{aligned}
log(E(Y_i)) - log(u_i) = log(\lambda_i) \\
log(E(Y_i)/ u_i) = log(\lambda_i)
\end{aligned}
\]

\end{frame}

\begin{frame}{Overdispersion}

\begin{itemize}
\item
  In Poisson models for \(Y_i \sim Poisson(\lambda_i)\) \[
  \begin{aligned}
  Var(Y_i) = \lambda_i
  \end{aligned}
  \]
\item
  In GLM estimation notation
\end{itemize}

\[
\begin{aligned}
\mu_i = E(\lambda_i) \\
Var(\mu_i) = \lambda_i
\end{aligned}
\]

\begin{itemize}
\tightlist
\item
  In an overdispersed model, the variance is higher because of some
  variability not captured by Poisson
\end{itemize}

\[
\begin{aligned}
Var(\mu_i) = \phi\lambda_i  \\ \phi > 0 
\end{aligned}
\] - Overdispersion implies \(\phi>1\)

\end{frame}

\begin{frame}{Overdispersion}

\begin{itemize}
\tightlist
\item
  Likelihood equations for Poisson GLM
\end{itemize}

\[
\begin{aligned}
\sum_{i=1}^{N} \frac{(y_i - u_i)x_{ij}}{Var(\mu_i)} \frac{\partial\mu_i}{\partial\eta_i} = 0 \\
j = 0, ..,p
\end{aligned}
\]

\begin{itemize}
\item
  Depends on the distribution of Y through \(\mu_i\) and Var(\(\mu\))
\item
  \(\phi\) drops out of the likelihood equations - this makes sense;
  variability won't affect the MLE - that is, \(\beta\)s are identical
  for models with \(\phi\) \textgreater{} 1 and \(\phi\) = 1
\item
  However, \(\phi\) does impact estimated standard errors
\end{itemize}

\end{frame}

\begin{frame}{Overdispersion}

\[
\begin{aligned}
w_i = (\frac{\partial u_i}{\partial \eta_i})^2 / Var(Y_i) \\
cov(\hat{\beta}) = (X^TWX)^{-1} = \phi cov(\hat{\beta})
\end{aligned}
\]

\begin{itemize}
\tightlist
\item
  \(\phi\) does not affect the \(\beta\)s but it does affect their
  covariance as a scaling factor
\end{itemize}

\end{frame}

\begin{frame}{Is Overdispersion Term Needed in a Model?}

\begin{itemize}
\item
  (See example 4.7.4 in Agresti)
\item
  Start with standardized residuals
\item
  Assume:\\
  \[
  \begin{aligned}
  z_i = \frac{y_i - \hat{y_i}}{\sqrt{Var(\hat{y_i})}} \\
  = \frac{y_i - \mu_i}{\sqrt{\mu_i}} \sim N(0,1) \\
  \sum_{i =1}^n z_i^2 \sim \chi^2_{n-k}
  \end{aligned}
  \]
\item
  where k is the number of parameters\\
\item
  if the sum of \(z_i^2\) is large (compare to chi-squared), we may need
  an overdispersion term \(\phi\)
\end{itemize}

\end{frame}

\begin{frame}{Is Overdispersion Term Needed in a Model?}

\[
\begin{aligned}
\hat{\phi} = \frac{\sum_{i=1}^n z_i^2}{n-k}
\end{aligned}
\]

\begin{itemize}
\item
  summarizes overdispersion in data compared to the fitted model
\item
  if \(\phi^2\) \textgreater{} 1, we should use the ``quasipoisson''
  family in R's glm() function
\item
  The SEs of a quasipoisson model are equivalent to the SEs of the
  Poisson model miultiplied by \(\sqrt{\hat{\phi}}\)
\end{itemize}

\[
\begin{aligned}
SE_{qp}(\hat{\beta}) = SE_p(\hat{\beta})*\sqrt{\hat{\phi}}
\end{aligned}
\]

\end{frame}

\end{document}
